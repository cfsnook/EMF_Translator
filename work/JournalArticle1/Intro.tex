\section{Introduction}
Event-B~\cite{ABR10} is one of a number of formal methods that may be used to model systems where a high degree of reliability is required. Event-B was inspired by its predecessor, \emph{Classical-B}~\cite{TheBBook}. It is a modelling language, used with a supporting tool platform, Rodin~\cite{abrial10rodin}; so named from the project in which it was developed~\cite{RodinTool}.  

In this section we introduce Event-B to the reader, and compare Event-B with some other formal approaches. We discuss automatic code generation from formal models, and potential target programming languages.

In Sect.~\ref{} we...
In Sect.~\ref{} we...

\subsection{Event-B}
The formal methods related to the work presented here can be categorized as state-based formal methods. Alternative, but not unrelated, approaches are categorized as process-based methods. Classical-B~\cite{TheBBook,CNP,CNPInterface,B4Free} and its successor, Event-B are said to be state-based, since they focus on modelling the changes of state, not the behaviour of processes. In Classical-B, state updates are modelled by guarded operations, where the operation is an analogue of a procedure call in a programming language. In Event-B, state updates are modelled by guarded events, providing a more abstract view of the way a system evolves. Event-B can be used to model systems at an abstract level; and by adding more detail (using a technique called refinement) it can model the software aspects of systems too. Both methods are set theoretic modelling approaches that incorporate a notion of proof to show that important system properties are maintained. The former is primarily an approach to software systems development, the latter more widely applicable to system-modelling. In an effort to make modelling and proof easier, Event-B was developed to overcome some of the difficulties encountered when using in Classical-B. The main differences between Classical and Event-B are highlighted in~\cite{Hallerstede07}, and inspiration was also drawn from action systems~\cite{Back1990133}.

It is fair to say that Event-B is not just a formal modelling language; the name is used to describe both a notation, and a methodology. In addition to this a mature tool-platform called \emph{Rodin}, named after its development programme, complements the methodology. The main modelling components of Event-B are contexts and machines. Contexts are used to model static features using sets, constants, and axioms. Machines are used to model variable state using \emph{variables}. A third, more recent addition, is the Theory component; where a developer can augment the bundled mathematical language, and rule-base, with new (inference and re-write) rules, data types, and operators. During the modelling process, changes to the components result in automatic generation of proof obligations, which must be discharged in order to show that the development is consistent. The proof obligations generated in classical-B are often complex, the Event-B approach results in simpler proof obligations as described in~\cite{Hallerstede07}, since Event-B consists of a simplified action syntax, giving rise to simpler proof obligations. A further simplification was made by adopting an event-based approach, where each atomic event has a predicate guard and an action consisting only of assignment statements. Events correspond to operations in the B-method; operation specification was more expressive, and included constructs for specifying operation preconditions (as part of its Design by Contract approach), operation calls, return parameters, and more complex structures for branching and looping. These constructs are not features of Event-B. Due to these simplifications (and more efficient proof tools) a large number of the proof obligations may be discharged automatically, by the automatic provers. Where un-discharged proof obligations remain, the user has, at their disposal, an interactive prover. Various techniques can be applied, to discharge the proof obligations, such as adding hypotheses; or making use of the hyperlink-driven user interface, for rule and tactic application. 

As we mentioned earlier in the section, Event-B makes use of a technique called refinement, where a machine can be refined by another. During this process new variables, events and invariant properties can be added; or existing events can be modified, but in a restricted manner. Machine refinement is transitive and leads to a hierarchical structure. Refinements are related to their more abstract counterparts in such a way that, a valid refinement always satisfies the specifications higher in the refinement hierarchy. In this way, important system properties can be specified at a high level of abstraction, and maintained down through the refinement chain. The Event-B tools are responsible for generating the proof obligations relating to refinement; these must be discharged in a similar way to those generated for proof of machine consistency. In some cases we may model entities in an abstraction that are defined in the event parameters; and in the refinement these entities may be introduced to the model as machine variables. To assist with the proof effort, we can link the parameters of the abstract event with their concrete counterpart using a $WITNESS$, this construct is a predicate describing the relationship between the event parameter of the abstraction and a corresponding variable in the refinement. It is often necessary to specify a linking invariant, to describe the relationship between the variables of the abstract and refinement machines. Inspection of the proof obligations can assist in this task since some of the un-discharged proof obligations provide information about this link. Another feature of Event-B is the ability to  refine one atomic event with a number of events, thus breaking the atomicity, as described in~\cite{Butler08}. Eventually, at the end of a refinement chain the models are detailed enough to accurately describe an implementation. But Event-B is a modelling language, and there is a disjunction between the description of the system in Event-B, and commonly used programming languages, such as Java, C and Ada. Addressing the semantic gap between Event-B and programming constructs is the subject of this article, which we will introduce fully, later in the article.

In terms of the recommended methodology, Event-B development begins with the abstraction, and modelling, of the observable events occurring in a system. Event-B (corresponding to its name) takes an event-based view of a system. The event-based approach uses guarded events to describe the observable events. An event is said to be enabled when the guard is true, and the state updates, described in the event actions, can take place; otherwise it is disabled, and none of its updates can occur.  An example of an Event-B machine can be seen in Fig.~\ref{fig:controllerSpec2}. It shows an abstract model of a pump controller, used in one of the case studies. We will use this model to describe some features of Event-B. But first we introduce the case study, which models a discrete \emph{pumpController}. The model describes a system where the controller receives a value of the fluid level, and a boolean value representing a user-request to turn the pump. Based on the inputs to the controller, a command to turn the pump on may be issued, or a warning is issued if a minimum level \emph{MIN} has been reached.    
%
%
%
\begin{figure}
\centering
\begin{minipage}{0.95\textwidth}
\textcolor{blue}{MACHINE} m1 \textcolor{blue}{REFINES} m0 \textcolor{blue}{SEES} ctx \\
\textcolor{blue}{VARIABLES} \text{m\_level, c\_level, e\_level, m\_pumpOnReq, c\_pumpOnReq, e\_pumpOnReq,} \hspace*{0.2cm} m\_pumpOnCmd, c\_pumpOnCmd, e\_pumpOnCmd, m\_warn, c\_warn, e\_warn,\\
\hspace*{0.2cm} c\_level\_internal, c\_pumpOnReq\_internal\\
\textcolor{blue}{INVARIANTS}\\
\hspace*{0.2cm}(c\_level\_internal $\leq$ MIN $\land$  c\_pumpOnReq\_internal = TRUE $\limp$  c\_warn = TRUE)\\
\hspace*{0.2cm} $\land$ (c\_level\_internal $>$  MIN $\land$  c\_pumpOnReq\_internal = TRUE\\
\hspace*{0.5cm} $\limp$  c\_pumpOnCmd = TRUE)\\
\hspace*{0.2cm} $\land$ (c\_level\_internal $\in  \intg$)\\
\hspace*{0.2cm} $\land$ (c\_pumpOnReq\_internal $\in$  BOOL) \ldots\\
\textcolor{blue}{EVENTS}\\
\textcolor{blue}{INITIALISATION} c\_level :=  100 $\pprod$ m\_level := 80 $\pprod$ c\_pumpOnReq :=  FALSE $\pprod$ \ldots\\
\textcolor{blue}{EVENT} fmiSetBoolean\_c \textcolor{blue}{REFINES} fmiSetBoolean\_c\\
\hspace*{0.2cm}\textcolor{blue}{ANY} p\\
\hspace*{0.2cm}\textcolor{blue}{WHERE} p = c\_compound $\land$ p $\in$ BOOL  \\
\hspace*{0.2cm}\textcolor{blue}{THEN} m\_pumpOnCmd :=  p\\
\hspace*{0.2cm}\textcolor{blue}{END}\\
\ldots
\end{minipage}
\caption{An Event-B  Pump Controller Model}
\label{fig:controllerSpec2}
\end{figure}
%
%
%
 In Fig.~\ref{fig:controllerSpec2}, we see that machine \emph{M1} refines another machine, \emph{M0}. It also has a \emph{SEES} clause, to make the contents of a context visible. The context may contain sets, constants, axioms and theorems. There are variables representing the internal state of the controller, and invariants providing type information for variables. Invariants are also used to describe the safety properties of the system. This describes a required safety property, that if the level is at or below \emph{MIN}, and a user's pump-on request is detected, then a warning will be issued. Also, if the level is OK and a pump-on is requested, then the state \emph{pumpOnCmd = TRUE} is  set.  Following the \emph{INVARIANTS} clause are the model's \emph{Events}. The \emph{Initialisation} event is special event, since it has no guards. The initialisation event of a machine must occur before any other event in the machine is enabled. The event in the figure has a parameter \emph{p}, in the \emph{ANY} clause. Parameters can be used to represent information flow, in and out of events, or they can represent a \emph{local} variable within the scope of the event. The event guard is defined in the \emph{WHERE} clause, in the example, where \emph{p} is typed as a Boolean. The guard relates the parameter to a machine variable \emph{c\_pumpOnCmd}, in the predicate $ p = c\_compound$. The event action appears in the \emph{THEN} clause, where the parameter is assigned to the variable \emph{m\_pumpOnCmd}, in the expression $m\_pumpOnCmd := p$.

---




-----
-----

%
%
%
One important aspect of the Event-B approach is that $any$ enabled event may occur, but only one of the enabled events may occur at any one moment. When modelling certain aspects of a system we may wish to impose an ordering of events. However, there is no sequence operator provided in the Event-B approach. It is therefore necessary to make appropriate use of guards and and state variables to model this aspect of a system. For example if we wish to impose an ordering on two events \emph{evt1} and \emph{evt2} so that $evt1$ occurs before $evt2$ we can use the following approach. Introduce an enumerated set $Grds = \{one,~ two,~ stop\}$ and a variable $step~\in~Grds$. Initially $step~\bcmeq~one$; and we make use of $step$ in the event guards as follows,
\begin{equation}
\begin{split}
&evt1 = \textbf{WHEN}~ step = one~ \textbf{THEN}~\ldots\pprod step \bcmeq two~ \textbf{END}\\
&evt2 = \textbf{WHEN}~ step = two~ \textbf{THEN}~\ldots\pprod step \bcmeq stop~ \textbf{END}\\
\end{split}
\notag
\end{equation}
This ensures that initially $evt1$ is enabled and $evt2$ is disabled since $step = one$; only after $evt1$ has updated the $step$ variable  to $two$ is $evt2$ enabled. At this time $evt1$ is no longer enabled since its guard is now false. Finally no events are enabled since $step = stop$ and all guards are false.


\subsection{Related Approaches}



-- VDM


-- Z
\subsection{Code Generation Rationale}


\subsection{Targets for Code Generation}

- Ada

- Java

- FMI-C
