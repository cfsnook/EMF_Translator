%\documentclass[handout]{beamer}
\documentclass{beamer}

\usepackage{bsymb}
\usepackage{b}
\usepackage{xcolor}


\mode<presentation>
{
    	%\usetheme{Warsaw}
	\setbeamertemplate{footline}
	{\centerline{\insertframenumber/\inserttotalframenumber}}
} 


\title{The Metro Refinement Example}

\author{Andy Edmunds}

\institute{ University of Southampton }



\begin{document}



\begin{frame}
\titlepage
\end{frame}

\begin{frame}
\frametitle{The Metro System}
\begin{itemize}
	\item Import the exercise model from MetroSafety\_Exercise.zip
	\item Read and understand, as much as possible, the variables, and events.
	\item Do not focus, to much, on the axioms, and invariants at this stage; but the following may be useful.
	\begin{itemize}
		\item $CDV$ is a (carrier) set of track sections.
		\item $cdvrel = CDV \rel CDV$
		\item $cdvfn = CDV \pfun CDV$
		\item $tcl \in cdvrel \tfun cdvrel$ is the transitive closure of track sections.
	\end{itemize}
	\item Add a message send event, indicating that a train may proceed to its next section. Introduce $tmsgs \in trns \tfun \pow(\Bool)$
	\item Model receipt of the permission by introducing an event, and $ permit \in trns \tfun \Bool $
	\item Modify \emph{enterCDV}, \emph{leaveCDV}, \emph{brake} (when a train's permit is false), and \emph{addTrain}.
\end{itemize}
\end{frame}





\end{document}




